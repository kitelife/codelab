\documentclass[11pt,a4paper]{article}
\usepackage{CJK} 
\usepackage{manfnt}   %为了特殊命令{\small\dbend}
\usepackage{verbatim} %提供多行注释环境,但是这种做法在数学环境等复杂环境中不起作用
\pagestyle{headings}
%\title{Algorithms Homework}
\begin{document}
\begin{CJK*}{UTF8}{gbsn}
%\maketitle
\begin{center}
\Large \bf Algorithms\ \ \ \ Homework
\end{center}
\begin{center}
\tt 学号:1100379058\ \ \ \ \ \ \ \ \ 姓名:夏永锋 
\end{center}
$\vbox{\hbox{\dbend}\vskip 11pt}$\newline
1.23 Show that if a has a multiplicative inverse modulo N,then this inverse is unique(modulo N).\newline\newline
证明:\ 假设a有两个multiplication inverse,则即存在x,y满足
\begin{equation}
ax\equiv ay\equiv 1 (mod\ N)
\end{equation}
\ \ \ \ \ \ \ \ \ \ \ \ \ \ \ \ \ \ \ 
\ \ \ \ \ \ $\Rightarrow$ 
\begin{equation}
axx\equiv axy (mod\ N)
\end{equation}
\ \ \ \ \ \ \ \ \ \ \ \ \ \ \ \ \ \ \ 
\ \ \ \ \ \ 则:
\begin{equation}
x\equiv y (mod\ N)
\end{equation}
\ \ \ \ \ \ \ \ \ \ \ \ \ \ \ \ \ \ \ 
\ \ \ \ \ \ 证毕!
\begin{flushleft}
$\vbox{\hbox{\dbend\dbend}\vskip 11pt}$
\end{flushleft}
1.31 Consider the problem of computing $N!=1\cdot2\cdot3\cdots N.$
\flushleft
\begin{enumerate}
	\begin{itemize}
		\item (a)\ If N is an n-bit number,how many bits long is N!, approximately(in $\Theta(\cdot)$ form)?
		\item (b)\ Give an algorithm to compute N! and analyze its running time.
	\end{itemize}
\end{enumerate}
\begin{enumerate}
解答:
	\begin{itemize}
		\item (a). 要想求$\Theta(\cdot)$, 可先求$O(\cdot)$和$\Omega(\cdot)$. 如果两者相等, 即可得到$\Theta(\cdot)$
		$\Rightarrow$\ \ \ \ \ \ N!的位数$=\log(N!)=O(\log(N^N))=O(N\log(N))$\newline
		另外:\ \ $\log(N!)=\Omega(\log(\frac{N}{2})^\frac{N}{2})=\Omega(N\log(N))$\newline
		同时:\ \ $n=\log(N)$\newline
		所以N!的位数为$\Theta(Nn)$
		\item (b). 先将N个数依次分为两个一组,利用AI Khwarizmi算法求出积,然后对结果两两分组,使用相同的算法进行计算,依次循环,直到得出最后的总乘积。\newline
		算法复杂度:由于AI Khwarizmi算法的总时间复杂度为$O(n^2)$, 与传统方法无异。\newline
		所以上述算法的时间复杂度为: \newline
		$O(\sum_{i=1}^{n}i\log i\times\log i)=O((\log n)^2\sum_{i=1}^{n}i)=O((n\log n)^2)$
	\end{itemize}
\end{enumerate}
\begin{flushleft}
$\vbox{\hbox{\dbend\dbend\dbend}\vskip 11pt}$
\end{flushleft}
1.35 Wilson's theorem says that a number N is prime if and only if\newline
\begin{center}
$(N-1)!\equiv -1\ (mod\ N)$
\end{center}
\flushleft
\begin{enumerate}
	\begin{itemize}
		\item(a).If p is a prime, then we know every number $1\leq x<p$ is invertible modulo p. Which of these numbers are their own inverse?
		\item(b).By pairing up multiplicative inverses, show that $(p-1)!\newline \equiv -1(mod\ p)$ for prime p.
		\item(c).Show that if N is not prime, then $(N-1)!\not \equiv -1(mod\ N)$\newline(Hint:Consider $d=gcd(N,(n-1)!)$.)
		\item(d).Unlike Fermat's Little theorem, Wilson's theorem is an if-and-only-if condition for primality. Why can't we immediately base a primality test on this rule?
	\end{itemize}
\end{enumerate}
\begin{enumerate}
解答:
	\begin{itemize}
		\item(a).由已知可得:
		\begin{displaymath}
			\left\{ \begin{array}{ll}
			x^{p-1}\equiv 1(mod\ p)\\
			x\cdot x\equiv 1(mod\ p)\\
		\end{array} \right.
		\end{displaymath}
显然1是, 另外, 由于$(p-1)=p(p-2)+1$ ,所以$p-1$也是。
		\item(b).$(p-1)!=1\times 2\times 3\times\cdots\times(p-2)\times(p-1)$,共$p-1$个因子相乘。显然地p是奇数,故这$p-1$个因子除了首尾的1和$p-1$之后,必然还剩下偶数个因子。设剩下的这偶数个因子的集合为S,如果S不为空,由互逆的对称性可以知道,S中的任何元素a的逆元($e^{-1}mod\ p$)也在S中,这样就构成了$\frac{\|S\|}{2}$个互逆对,所以S中所有元素的乘积模p为 1,于是有:
$(p-1)!=1\times 2\times 3\times\cdots\times(p-2)\times(p-1)\newline
=1\times (p-1)\newline
=-1 (mod\ p)$
		\item(c).运用\textbf{反证法}:\newline
		若$(N-1)!\equiv -1(mod\ N)$, 则有$(N-1)!\times (-1)\equiv 1(mod\ N)$, 即$(N-1)!$存在逆元.这需要$gcd((N-1)!,N)=1$, 而如果N不为素数,则$gcd((N-1)!,N)$必不大于1,得出矛盾,证明完毕。
		\item(d). 设N有n bits,费马小定理测试一个数是否为素数只需要计算$a^{N-1}$,因此只需要$O(n)$次乘法,但使用Wilson定理,则需要$O(2^n)$次,相比较而言,Wilson定理的效率太低,所以一般情况下不使用。
	\end{itemize}
\end{enumerate}


\begin{flushright}
2010-3-10
\end{flushright}
\end{CJK*} \end{document}