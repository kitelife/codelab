\documentclass[12pt]{article}

\usepackage{CJK}
\usepackage{pslatex,color}
\usepackage[margin=2cm]{geometry}

\begin{document}
\begin{CJK*}{UTF8}{gkai}

\title{\color{blue}关于成立"众成技术聚乐部"的倡议}
\author{夏永锋}
\date{}
\maketitle

\newpage
\section*{理念}
目前开源社区类似的俱乐部、交流活动不少,但内容都偏向浮泛,甚至成为了赞助商的广告平台。
对于多数程序员自身技术能力的提高来说,更重要的是围绕技术主题,系统深入地学习讨论。计算机技术发展至今日,分门别类非常多,仅靠一人之力,要想在广度深度上都有较好的掌握,实非易事,并且技术这条路很寂寞很辛苦,志同道合的技术爱好者应该一起努力,一起来学习分享,共同进步。

\section*{运作}
\begin{enumerate}
\item 为保证技术交流的质量,众成技术聚乐部的交流活动并不是任何人都能参加,参加的人必须是聚乐部的成员。非聚乐部成员可以通过申请或聚乐部成员推荐,经资格审核通过后,成为聚乐部成员
\item 聚乐部成员各自提交技术主题列表,共同讨论确定半年/一年内待交流探讨的技术主题,然后一人或多人各自认领一个主题研习准备
\item 交流每半个月/一个月一次,每次交流内容分两部分:主题交流、尽兴分享。尽兴分享题目需在分享日期的5天前提交,内容主要涉及工作中、业余项目中遇到的问题及其解决方案等
\item 建立聚乐部Github组织帐号,所有分享的幻灯片、代码都提交到该帐号下指定的代码库。聚乐部成员协作开发的项目也放在该帐号之下
\item 建立聚乐部微博等社交帐号,对外宣传聚乐部活动,吸引更多志同道合的技术爱好者来申请成为聚乐部成员
\item 聚乐部活动需要的场地暂由XXX公司提供;某些聚乐部成员能够提供场地的话,也可以选择
\item 聚乐部活动可能产生的费用开销,由聚乐部成员共同承担
\end{enumerate}

\section*{问题}
\begin{enumerate}
\item 是否可以邀请非聚乐部成员的业界技术大牛来做技术分享?
\item 做分享的成员临时有事,如何处理?(所有分享提前一期准备好?)
\end{enumerate}

\end{CJK*}
\end{document}