%% start of file `template-zh.tex'.
%% Copyright 2006-2013 Xavier Danaux (xdanaux@gmail.com).
%
% This work may be distributed and/or modified under the
% conditions of the LaTeX Project Public License version 1.3c,
% available at http://www.latex-project.org/lppl/.


\documentclass[12pt,a4paper,sans]{moderncv}   % possible options include font size ('10pt', '11pt' and '12pt'), paper size ('a4paper', 'letterpaper', 'a5paper', 'legalpaper', 'executivepaper' and 'landscape') and font family ('sans' and 'roman')

% moderncv 主题
\moderncvstyle{banking}                        % 选项参数是 ‘casual’, ‘classic’, ‘oldstyle’ 和 ’banking’
\moderncvcolor{blue}                          % 选项参数是 ‘blue’ (默认)、‘orange’、‘green’、‘red’、‘purple’ 和 ‘grey’
%\nopagenumbers{}                             % 消除注释以取消自动页码生成功能

% 字符编码
\usepackage[utf8]{inputenc}                   % 替换你正在使用的编码
\usepackage{CJKutf8}

\usepackage{tikz}

\usepackage[lf]{venturis} %% lf option gives lining figures as default;
			  %% remove option to get oldstyle figures as default
%\renewcommand*\familydefault{\sfdefault} %% Only if the base font of the document is to be sans serif
\usepackage[T1]{fontenc}

% 调整页面
\usepackage[scale=0.8]{geometry}
%\setlength{\hintscolumnwidth}{3cm}           % 如果你希望改变日期栏的宽度

% 个人信息
\name{夏永锋}{}
\title{个人简历}                     % 可选项、如不需要可删除本行
\phone[mobile]{15921584916}              % 可选项、如不需要可删除本行
\email{youngsterxiayf@gmail.com}                    % 可选项、如不需要可删除本行
\homepage{blog.xiayf.cn}            % 可选项、如不需要可删除本行
\extrainfo{\url{github.com/youngsterxyf}}
%----------------------------------------------------------------------------------
%            内容
%----------------------------------------------------------------------------------
\begin{document}
\begin{CJK}{UTF8}{gkai}                       % 详情参阅CJK文件包
\maketitle


\section{工作背景}

\cventry{2014.05 -- 至今}{资深研发工程师}{百度}{上海}{}{}
\begin{itemize}
	\item {\color{blue}\href{https://cloud.baidu.com}{百度云 \tikz \draw[->, thick](0, 0) cos(0.1,0.2);}}
	\begin{itemize}
		\item 个人职责:负责风控相关核心系统的架构设计与研发
		\item 相关技术:Spark、Kafka、Hadoop、Zookeeper、Mongodb、Spring 等
	\end{itemize}
	\item {\color{blue}\href{https://bsi.baidu.com}{百度安全指数 \tikz \draw[->, thick](0, 0) cos(0.1,0.2);}}
	\begin{itemize}
		\item 简介:基于一套多维度(实时安全、历史安全、网站环境、攻击风险等)标准模型,对站点的整体情况进行评估打分,并提供各维度评分的详细依据;此外还提供互联网整体及分行业的安全评估
		\item 个人职责:
		\begin{itemize}
			\item 负责产品架构设计,独立完成 Web 后端开发
			\item 参与 Spark 大数据处理开发工作
			\item 负责核心资产漏洞库API、指纹识别基础服务等开发工作
			\item 负责相关产品的业务运维、可用性保障
		\end{itemize}
		\item 相关技术:Go、Java、Spark、Python等
	\end{itemize}
	\item {\color{blue}\href{http://ce.baidu.com}{百度云观测 \tikz \draw[->, thick] (0,0) cos(0.1,0.2);}}
	\begin{itemize}
		\item 简介:为中小站点提供可用性、访问速度、安全等方面的监测服务
		\item 个人职责:Web 后端功能开发,开放 API 设计与实现,微信服务号开发与维护 等
		\item 相关技术:PHP、Yii、Python、Go 等
	\end{itemize}
\end{itemize}

\cventry{2013.04 -- 2014.05}{运营开发工程师}{腾讯}{上海}{}{}
\begin{itemize}
	\item 内部运营平台
	\begin{itemize}
		\item 简介:针对易迅业务的线上服务器、网络设备,通过指标数据的采集、展示、告警等进行服务监控
		\item 个人职责:
		\begin{itemize}
			\item 服务器信息配置模块:为所有告警模块、数据图形化展示提供基础数据
			\item 服务器归类模块:方便分类展示、检索服务器
			\item URL监控:见{\color{blue}\href{http://blog.xiayf.cn/2013/10/12/url-monitoring-and-web-arch/}{文章}}说明
		\end{itemize}
		\item 相关技术:PHP、Redis、Memcached、Bootstrap、Highcharts、D3.js 等
	\end{itemize}

	\item {\color{blue}\href{http://blog.xiayf.cn/2013/10/16/high-availability-load-balancer-and-dns/}{高可用容灾方案 \tikz \draw[->, thick] (0,0) cos(0.1,0.2);}}
	\begin{itemize}
		\item 简介:主要针对易迅 IDC 与腾讯 IDC 之间的互通,提供高可用容灾方案
		\item 个人职责:
		\begin{itemize}
			\item 基于 HAProxy 和 Keepalived 提供网络双链路的负载均衡与互备
			\item 基于 BIND 为 IDC 内部搭建缓存 DNS
			\item 使用 Go 语言实现 HAProxy 负载均衡任务管理系统 {\color{blue}\href{https://github.com/youngsterxyf/haproxyconsole}{HAProxyConsole \tikz \draw[->, thick] (0,0) cos(0.1,0.2);}}
		\end{itemize}
		\item 说明:该项目由个人独立完成;HAProxyConsole 在腾讯其他部门也得到应用。
	\end{itemize}

	\item {\color{blue}\href{http://blog.xiayf.cn/2013/11/29/inner_warehouse_monitor_system/}{易迅全国仓库作业机器监控系统 \tikz \draw[->, thick] (0,0) cos(0.1,0.2);}}
	\begin{itemize}
		\item 简介:针对易迅业务全国仓库内的普通作业 PC,采集、存储、展示 CPU 使用率、内存使用率、网卡流量、开关机状态等指标数据,以辅助机器故障分析
		\item 个人职责:系统的总体设计,技术选型,NSQ 客户端实现(包含数据存储),数据的展示等
		\item 相关技术:Go、Beego、NSQ、Saltstack 等
	\end{itemize}
\end{itemize}

\cventry{2012.05 -- 2012.12}{Web 开发实习生}{谷歌-企业社会责任部}{上海}{}{}
\begin{itemize}
	\item 工作简介:负责开发维护{\color{blue}\href{www.17gong1.com}{一起公益网}}、{\color{blue}\href{www.gong1pin.com}{公益品网}} 等
	\item 相关技术:CentOS、Nginx、PHP、MySQL 等
\end{itemize}


\section{业余项目}

\begin{itemize}
	\item{\color{blue}\href{http://beego.me/}{Beego - Go Web框架 \tikz \draw[->, thick] (0,0) cos(0.1,0.2);}}
	\begin{itemize}
		\item Star数:1.4万+
		\item 职责:核心开发者
	\end{itemize}
	\item{\color{blue}\href{https://github.com/youngsterxyf/redis-sentinel-ui}{redis-sentinel-ui \tikz \draw[->, thick] (0,0) cos(0.1,0.2);}}
	\begin{itemize}
		\item: 简介:Redis Sentinel 集群监控后台,提供集群运行概况、监控数据可视化、数据查询功能
	\end{itemize}
	\item{\color{blue}\href{https://github.com/youngsterxyf/memcached-ui}{memcached-ui \tikz \draw[->, thick] (0,0) cos(0.1,0.2);}}
	\begin{itemize}
		\item: 简介:Memcached 集群 Web UI,提供集群运行概况、在线操作功能,内置 memcache 协议实现
	\end{itemize}
	\item 微信公众号 “i练” Web APP
	\begin{itemize}
		\item 相关技术:单页面应用,Framework7、require.js、微信 JS SDK 等
	\end{itemize}
	\item 佑擎财富管理有限公司ERP系统
	\begin{itemize}
		\item 相关技术:Python、Django、Celery、Bootstrap、Highcharts、QQ企业邮箱无缝集成 等
	\end{itemize}
	\item {\color{blue}\href{https://github.com/youngsterxyf/fuse}{支持多平台的 Git webhook 服务 \tikz \draw[->, thick] (0,0) cos(0.1,0.2);}}
	\begin{itemize}
		\item 简介:针对 Git 多分支工作流模型,以插件化方式实现多平台 Webhook 支持,配置灵活
		\item 相关技术:Go、Martini、FlatUI 等
	\end{itemize}
	\item {\color{blue}\href{https://github.com/youngsterxyf/feed-world}{Feed 聚合服务 \tikz \draw[->, thick] (0,0) cos(0.1,0.2);}}
	\begin{itemize}
		\item 相关技术:Slim、Vue.js、Github 及微博 OAuth2 登陆
	\end{itemize}
\end{itemize}


\section{教育背景}
\cventry{2010.9 -- 2013.4}{硕士}{上海交通大学 - 软件学院}{上海}{}{}  % 第3到第6编码可留白
\cventry{2006.9 -- 2010.7}{学士}{上海大学 - 计算机工程与科学学院}{上海}{}{}


\section{语言技能}

\cvitemwithcomment{英语}{CET-6}{读、写 - 流畅,听、说 - 一般}
\begin{itemize}
\item 翻译出版技术书籍《精通 Python 设计模式》、《Python 漫游指南》,经常翻译优秀技术文章(见{\color{blue}\href{http://blog.xiayf.cn}{个人博客}})
\end{itemize}

\clearpage\end{CJK}
\end{document}

%% 文件结尾 `template-zh.tex'.
