\documentclass{beamer}
%This is the file main.tex
\usepackage[english]{babel}
\usetheme{Berlin}

\title{Example Presentation Created with the Beamer Package}
\author{Till Tantau}
\date{\today}

\begin{document}

\begin{frame}
    \titlepage
\end{frame}

\section*{Outline}
\begin{frame}
    \tableofcontents
\end{frame}

\section{Introduction}
\subsection{Overview of the Beamer Class}
\subsection{Overview of Similar Classes}

\section{Usage}
\subsection{...}
\subsection{...}

\section{Examples}
\subsection{...}
\subsection{...}

% The empty frame at the end (which should be deleted later) ensures that the sections and subsections are actually part of the table of contents. This frame is necessary since a \section or \subsection command following the last page of a document has no effect. 
\begin{frame}
\end{frame} %to enforce entries in the table of contents

\end{document}
